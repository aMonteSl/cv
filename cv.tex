\documentclass[10pt,a4paper]{article}

% --- Paquetes básicos ---
\usepackage[spanish]{babel}
\usepackage[utf8]{inputenc}   % Con XeLaTeX puedes omitirlo
\usepackage[T1]{fontenc}
\usepackage{lmodern}
\usepackage[margin=1.55cm]{geometry} % un poco más de área útil
\usepackage{parskip}
\setlength{\parskip}{1.5pt}          % ligero pero legible
\usepackage{microtype}
\linespread{0.965}                    % pelín menos de interlínea
\usepackage{hyperref}
\hypersetup{colorlinks=false}
\urlstyle{same}
\usepackage[none]{hyphenat}
\sloppy % (opcional) da margen para evitar cortes agresivos


% --- Estilos del CV ---
\newcommand{\Name}[1]{%
	\begin{center}
		{\bfseries\fontsize{21}{23}\selectfont #1}\par
	\end{center}
}
\newcommand{\Contact}[1]{%
	\begin{center}
		\footnotesize #1
	\end{center}
}

% Título de sección: ROMANA + MAYÚSCULAS + línea
\newcommand{\SectionTitle}[1]{%
	\vspace{5pt}%
	{\bfseries\MakeUppercase{#1}}\par
	\rule{\textwidth}{0.5pt}\vspace{3pt}%
}

% Tablas compactas (educación / eventos / idiomas)
\newcommand{\TightTwoCol}[4]{%
	\begingroup\setlength{\tabcolsep}{0pt}\renewcommand{\arraystretch}{0.95}%
	\noindent\begin{tabular*}{\textwidth}{@{}l@{\extracolsep{\fill}}r@{}}
		\textbf{#1} & #3 \\
		\textit{#2} & \textit{#4} \\
	\end{tabular*}\par\endgroup
}

% Educación
\newcommand{\EduEntry}[4]{%
	\TightTwoCol{#1}{#2}{#3}{#4}\vspace{3pt}
}

% Hitos/Proyectos (viñeta + título en negrita)
\newcommand{\ProjEntry}[3]{%
	\noindent\textbullet\ \textbf{#1}\par
	\textit{#2}\par
	#3\par\vspace{3pt}%
}

% Eventos (viñeta)
\newcommand{\EventEntry}[4]{%
	\begingroup\setlength{\tabcolsep}{0pt}\renewcommand{\arraystretch}{0.95}%
	\noindent\begin{tabular*}{\textwidth}{@{}l@{\extracolsep{\fill}}r@{}}
		\textbullet\ \textbf{#1} & #3 \\
		\textit{#2} & \textit{#4} \\
	\end{tabular*}\par\endgroup
}
\newcommand{\EventDesc}[1]{%
	#1\par\vspace{1pt}%
}

% Habilidades: viñeta + título + lista separada por comas
\newcommand{\SkillEntry}[2]{%
	\noindent\textbullet\ \textbf{#1}: #2\par\vspace{0pt}%
}

% Idiomas (si necesitas filas)
\newcommand{\LangEntry}[2]{%
	\begingroup\setlength{\tabcolsep}{0pt}\renewcommand{\arraystretch}{0.95}%
	\noindent\begin{tabular*}{\textwidth}{@{}l@{\extracolsep{\fill}}r@{}}
		\textbf{#1} & #2 \\
	\end{tabular*}\par\endgroup
}

\begin{document}
	
	% --- Nombre ---
	\Name{ADRIÁN MONTES}
	
	% --- Contacto ---
	\Contact{%
		Email: \href{mailto:adrian.adyra@gmail.com}{adrian.adyra@gmail.com} \enspace|\enspace
		Tel: +34 637 682 361 \enspace|\enspace
		LinkedIn: \href{https://linkedin.com/in/adrianmonteslinares}{linkedin.com/in/adrianmonteslinares} \enspace|\enspace
		GitHub: \href{https://github.com/aMonteSl}{github.com/aMonteSl} \enspace|\enspace
		Madrid, ES
	}
	
	% --- RESUMEN ---
	\SectionTitle{Resumen}
	
	Estudiante de \textbf{Ingeniería Telemática} (URJC) con entusiasmo por iniciar mi carrera en \textbf{desarrollo de software} (\textbf{backend} y \textbf{frontend}). Busco \textbf{prácticas} para seguir creciendo técnicamente mientras aporto \textbf{valor} en \textbf{soluciones reales} y me comprometo con los objetivos del \textbf{equipo}.
	
	Mi proyecto más significativo es \textbf{Code\mbox{-}XR}, un plugin de \textbf{software libre} para \textbf{Visual Studio Code}. Me permitió consolidar \textbf{habilidades técnicas} y desarrollar \textbf{competencias transversales} como comunicación, difusión y autogestión.
	
	Intereses: \textbf{viajar} para conocer otras culturas y formas de pensar; explorar \textbf{nuevas tecnologías}; y mantener un \textbf{aprendizaje continuo}, entre otros intereses más mundanos. Si tuviera que resumir mi hobby en una palabra: \textbf{aprender}.
	
	% --- EDUCACIÓN ---
	\SectionTitle{Educación}
	\EduEntry{Universidad Rey Juan Carlos (URJC)}{Grado en Ingeniería Telemática}{Madrid, España}{2020--Presente}
	
	% --- HITOS Y PROYECTOS RELEVANTES ---
	\SectionTitle{Hitos y proyectos relevantes}
	\ProjEntry
	{Artículo científico (VISSOFT/ICSME)}
	{Creador de Code-XR como TFG (2025)}
	{Desarrollo de \textbf{Code-XR}, un plugin para \textbf{Visual Studio Code} que permite la visualización
		de métricas de código en tiempo real, con integración en \textbf{realidad virtual y aumentada}
		(\textit{A-Frame/WebXR}).}
	\ProjEntry
	{Matrícula de Honor — LSMU}
	{Desarrollo de apps Android con Kotlin (2025)}
	{Aplicación de principios de diseño de interfaces, gestión del ciclo de vida de actividades y uso avanzado de
		componentes del sistema en \textbf{Android Studio}.}
	\ProjEntry
	{Matrícula de Honor — AST}
	{Desarrollo de apps telemáticas en C++ (2024)}
	{Asignatura centrada en \textbf{programación orientada a objetos} en \textbf{C++}. Desarrollo de aplicaciones telemáticas
		aplicando técnicas y tecnologías de comunicación y diseño de software.}
	
	% --- ACTIVIDADES Y EVENTOS ---
	\SectionTitle{Actividades y eventos}
	\EventEntry{Hackatón URJC 2024}{Participante}{Madrid, España}{2024}
	\EventDesc{Hackatón de 3 días (economía circular). Modelo de \textit{tokens} universitarios: ganar por contribuir (tutorías, apuntes, actividades) y gastar en ayuda/servicios del campus. Rol: ideación, prototipado y \textit{pitch}.}
	\EventEntry{Charlas y cursos tecnológicos (online/presencial)}{Asistente}{Varias instituciones}{Desde 2023}
	\EventDesc{Participación regular en charlas y talleres sobre tecnologías emergentes y herramientas existentes (web/XR, backend, IA generativa).}
	
	% --- HABILIDADES ---
	\SectionTitle{Habilidades}
	\SkillEntry{Lenguajes}{Pascal, C, C++, Ensamblador (RISC\mbox{-}V), Python, Kotlin, JavaScript, TypeScript, HTML, CSS, MATLAB, SQL, PostgreSQL}
	\SkillEntry{Frameworks, librerías y XR}{Django, Bootstrap, A-Frame (WebXR), ROS~2}
	\SkillEntry{Herramientas y sistemas}{Git, Linux,VS Code, Android Studio, Wireshark/nmap, OpenSSL/GPG, iptables/UFW}
	\SkillEntry{Otros}{Nociones de \textit{prompt engineering} con IA generativa}
	
	% --- IDIOMAS ---
	\SectionTitle{Idiomas}\vspace{0pt}
	\noindent \textbf{Inglés}: B2 -- TOEIC 2022 \;·\; \textbf{Español}: Nativo
	
	% --- RECOMENDACIONES ---
	\SectionTitle{Recomendaciones}\vspace{-3pt}
	{\footnotesize\noindent Carta de recomendación \textit{disponible a petición} (Prof.\ David Moreno Lumbreras, URJC).}
	
\end{document}
